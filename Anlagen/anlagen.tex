\addchap{Anlagen}
\noindent\textbf{Anlage 1}
{\rowcolors{2}{white}{blue!30!white!70}
\begin{table}[h!]
  \begin{center}
    \caption{Zeitdifferenz zwischen GraphQL und REST}\label{tab:graphql-rest}
    \begin{tabularx}{\linewidth}{Xrr}
      \rowcolor{blue!40!white!50!black!10}\textbf{Szenario} & \textbf{relativ} & \textbf{absolut (in ms)}\\
      \midrule
      Data in Memory | Server Heroku & 33\% & 38,81\\
      DB Cloud | Server Heroku & 23\% & 83,38\\
      DB Cloud | Server Lokal & 21\% & 41,16\\
      \bottomrule
      arithmetisches Mittel der Klassen & 26\% & 54,45\\
    \end{tabularx}
  \end{center}
\end{table}
}
\par
\vspace{1cm}
\noindent\textbf{Anlage 2}
Alle Messungen sind auf dem beigelegten Datenträger in einer Excel-Datei zu finden.
\par
\vspace{1cm}
\noindent\textbf{Anlage 3}
Der Quellcode zur für den Vergleich verwendeten Software (VueJS-Client, REST-Server, GraphQL-Server) ist auf dem beigelegten Datenträger zu finden.