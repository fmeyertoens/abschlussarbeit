\chapter{GraphQL}

\section{Entwicklung und Grundgedanke}
\begin{itemize}
  \item GraphQL ist Abfragesprache für APIs und Laufzeitumgebung um auf Abfragen zu antworten
  \item Server stellt Schema = komplette Beschreibung der Datenstruktur
  \item Client sendet beliebige Abfrage und erhält exakt die angefragten Daten
  \item Eine Anfrage für alle nötigen Daten (für eine View, UI basiert); Vorteil bei langsamen mobilen Netzwerken
  \item 2012 von Facebook entwickelt und eingesetzt
  \item 2015 open source, GraphQL Foundation, Spec auf Github weiterentwickelt, Juni 2018 letzter Release
  \item Besteht aus Typsystem, Abfragesprache, Ausführungssemantik, statischer Validierung und Typintrospektion
\end{itemize}

\section{Spezifikation und Funktionsweise}
\begin{itemize}
  \item Typsystem
  \begin{itemize}
    \item Typsystem und GraphQL Schema drücken aus, welche Objekte die API zu Verfügung stellt
    \item Schema besteht aus `type', `enum' und `interface'. `type' kann `interface' implementieren
    \item jeder Type (und Interface) ist Ansammlung von Feldern
    \item `null' ist erlaubter Wert. Non-nullable Feld wird mit \enquote{!} markiert
    \item Einstiegspunkt (Top level) in Typsystem ist Objekttyp, Name nach Konvention `query'
    \item Felder auf query Typ sind mögliche Operationen; Argumente möglich
  \end{itemize}
  \item Query Syntax
  \begin{itemize}
    \item Abbildung Beispiel Query
    \item GraphQL Abfrage beschreibt deklarativ welche Daten erwartet werden
    \item Antwort ist JSON mit der gleichen Struktur
    \item Abfragen können geschachtelt werden
    \item Fragmente
    \begin{itemize}
      \item Abbildung Beispiel Fragment
      \item verhindert Dopplung von mehreren Feldern in Abfrage
      \item ermöglicht typbasierte Feldselektion
    \end{itemize}
  \end{itemize}
  \item Introspektion
  \begin{itemize}
    \item Spezialfelder beginnend mit doppelt Unterstrich
    \item \_\_schema, \_\_typename
    \item Metadaten über GraphQL Schemas
    \item Sinn ist Nutzung durch Entwicklungstools
    \item ermöglicht statische Validierung: GraphQL Abfrage kann zu Entwicklungszeit geprüft werden
  \end{itemize}
  \item Referenzen werden unsichtbar, da von GraphQL Server automatisch aufgelöst (Performance beachten!)
  \item ein Endpunkt (kein Nutzen von URIs)
  \item jede Abfrage mit POST (kein Nutzen von HTTP Methoden)
  \item Semantik der Abfrage von Server ausgewertet (welche der CRUD Operationen)
\end{itemize}

\section{Server-Execution}
