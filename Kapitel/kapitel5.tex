\section{Vergleich}

\subsection{Versionierung und Weiterentwicklung}
\begin{itemize}
  \item GraphQL
  \begin{itemize}
    \item neue Felder hinzufügen ohne existierende Queries zu beeinflussen
    \item neue Felder nicht automatisch gesendet
    \item Felder als deprecated markieren → Tool kann Entwickler warnen
    \item Monitoring auf Feldlevel möglich
  \end{itemize}
  \item REST
  \begin{itemize}
    \item Monitoring: werden sparse fieldsets oder includes verwendet können genutzte Felder aufgezeichnet werden
    \item neue Version, neuer Endpunkt example.com/v2/contacts/\dots
    \item für kleine Veränderungen ungeeignet
  \end{itemize}
\end{itemize}

\subsection{Datenaufkommen/Netzwerklast}
\begin{itemize}
  \item Transfer, Verarbeiten und Speichern unnötiger Daten (Felder) sollte vermieden werden
  \item GraphQL
  \begin{itemize}
    \item automatisch kleinstmöglicher Request
    \item Query muss an Server gesendet werden
  \end{itemize}
  \item REST
  \begin{itemize}
    \item standardmäßig gesamte Repräsentation
    \item viele APIs bieten Feldselektion an (Partials)
    \item query string enthält Feldselektion nach bestimmter Syntax
    \item je komplexer, desto mehr Daten
    \item jede Ressource ist extra Endpunkt
    \item für mehrere Ressourcen müssen mehrere Requests gemacht werden, n+1 Problem
    \item includes beziehen verbundene Daten in Response ein → ein Request für mehrere Ressourcen
  \end{itemize}
  \item GraphQL und REST Partials unterscheiden Objekt und Array nicht → Wissen über API notwendig um Performance einzuschätzen
  \item mehr Daten um Request genauer zu machen, sinnvoll um deutlich weniger Daten als Response zu erhalten
\end{itemize}

\subsection{Caching}
\begin{itemize}
  \item Flexibilität gegen Caching: je spezieller die Abfrage, desto schwieriger (weniger sinnvoll) caching
  \item nicht Antwort direkt cachen (HTTP), sondern manuell Objekte anhand ID cachen (JavaScript)
  \item REST
  \begin{itemize}
    \item Browser HTTP caching automatisch genutzt
    \item Caching basierend auf Endpunkt
    \item URL ist cache ID für die Ressource
    \item ermöglicht HTTP cache proxies
    \item je spezieller der query string, desto weniger cache Treffer
  \end{itemize}
  \item GraphQL
  \begin{itemize}
    \item POST response wird normalerweise nicht gecacht
    \item standardmäßig keine ID für caching vorhanden. Empfehlung: API sollte ID pro Objekt bereitstellen
  \end{itemize}
\end{itemize}

\subsection{Abfrageflexibilität}
\begin{itemize}
  \item GraphQL
  \begin{itemize}
    \item keine Wildcards (alle Felder eines Objekts)
    \item als Abfragesprache gedacht
    \item unterscheidet Feldselektion, Sortieren, Filtern, Optionen
    \item keine Syntax für Sortieren/Filtern, aber Möglichkeit über Feldargumente
    \item Bsp: height (unit:FOOT)
    \item sehr anpassbar für verschiedene Apps bzw. öffentliche API (Anforderungen unbekannt)
    \item include ist quasi Pflicht, effizient und konsistent
    \item an Query sind Performanceprobleme evtl.\ schwer erkennbar (siehe Tools)
  \end{itemize}
  \item REST
  \begin{itemize}
    \item oft einfacher Anfang. Mit steigender Komplexität werden Abfragesprachen typische Konstrukte eingebaut
    \item fields=`` '' für Feldselektion
    \item field=`` '' zum Filtern
    \item sort=[field],sort-dir=desc zum Sortieren
    \item option=`` '' für Optionen
    \item includes empfohlen bei JSON:API, aufbrechen von HATEOAS\@?
    \item includes oder extra Endpunkt ist Entwurfsentscheidung
    \item Anforderungen für Client App können spezifisch werden, extra API pro Client
  \end{itemize}
  \item Upload ist schwierig für GraphQL (serialization), REST kann multipart/form-data header nutzen
\end{itemize}

\subsection{Batching/Deduping}

\subsection{Fehlerbehandlung}

\subsection{Sicherheit}

\subsection{Kosten}

\subsection{Lernkurve, Fehlersuche, Community}

\subsection{Bibliotheken und Tools}
\begin{itemize}
  \item GraphQL
  \begin{itemize}
    \item offizielle Spezifikation vorhanden
    \item Referenzimplementierung in JavaScript
    \item Zusatztools von Facebook (Dataloader, Relay)
    \item Tool kann Abfragekomplexität zu Entwicklungszeit ermitteln (mit vergangenen Messwerten)
  \end{itemize}
\end{itemize}
