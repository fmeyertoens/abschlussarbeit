\chapter{Das REST Architekturkonzept}
\section{Entstehung}
\begin{itemize}
  \item Bekanntmachung Roy T. Fielding in Dissertation 2000
  \item Akronym für Representational State Transfer
  \item Prinzipien für die Entwicklung von verteilten Systemen. Baut auf bekannten Architekturen auf (Client-Server)
  \item REST hatte Einfluss (und macht Gebrauch von) HTTP und URI (IRI)
\end{itemize}
Das Architekturkonzept des Representational State Transfer wurde erstmals im Jahr 2000 von Roy T. Fielding erwähnt, definiert und mit dem Akronym REST bezeichnet.\cite[76]{REST}
Es enthält eine Reihe von Prinzipien für die Entwicklung von verteilten Systemen im Web mit besonderem Fokus auf den Schnittstellen zwischen den einzelnen Elementen.
Dabei baut REST auf der Client-Server-Architektur auf und hatte Einfluss auf die Entwicklung von HTTP und URI, wodurch jene Schnittstellen im Web definiert werden.\cite[vgl.][4,105f.]{REST}
REST beschreibt größtenteils die Architektur des Web selbst und ist daher nicht auf APIs ausgelegt oder beschränkt.
Durch die Arbeit von Fielding in den HTTP- und URI-Arbeitsgruppen hatten seine Gedanken zu REST direkten Einfluss in die Entwicklung der beiden Standards.\cite[107]{REST}

\section{Grundlagen}
\begin{itemize}
  \item Ziele von REST
  \begin{itemize}
    \item skalierbare Komponenteninteraktionen
    \item generische Interfaces
    \item unabhängige Entwicklung der Komponenten
    \item Zwischenkomponenten können spezielle Aufgaben übernehmen (Cache, Sicherheit, Schnittstelle zu Altsystemen)
  \end{itemize}
\end{itemize}
\subsubsection{Ziele von REST}
Die Ziele und Vorteile einer Anwendung auf Basis von REST lassen sich folgendermaßen zusammenfassen:
\foreignblockcquote{english}[105]{REST}{REST provides a set of architectural constraints that, when applied as a whole, emphasizes scalability of component interactions, generality of interfaces, independent deployment of components, and intermediary components to reduce interaction latency, enforce security, and encapsulate  legacy  systems.}
Diese Ziele werden erreicht durch eine Reihe von Anforderungen.
\begin{itemize}
  \item REST ist zustandslos (daher auch client-stateless-server). Jeder Request von Client zu Server beinhaltet alle notwendigen Informationen, um den Request zu verstehen.
  \item Client speichert gewöhnlich Zustand. Server behält keine Clientsession bei.
  \item Vorteile
  \begin{itemize}
    \item Sichtbarkeit: Requests können einzeln untersucht werden
    \item Zuverlässigkeit: einfachere Wiederherstellung bei teilweisen Fehlern
    \item Skalierbarkeit: Server kann schnell Ressourcen wieder freigeben. Speichert keine Zustände
  \end{itemize}
  \item Nachteile
  \begin{itemize}
    \item verringerte Netzwerkperformance, da mit jedem Request Daten wiederholt
    \item verringerte Kontrolle des Servers über Verhalten der Clientanwendung
  \end{itemize}
  \item Clientseitiges Caching
  \begin{itemize}
    \item Ressourcen implizit oder explizit gecacht
    \item erlaubt Wiederverwendung früherer Serverantworten für zukünftige, gleiche Requests
    \item bessere Effizienz und Skalierbarkeit, erhöhte gefühlte Performance durch Verringerung der durchschnittlichen Latenz (jede einzelne Latenz durch Cache-Lookup erhöht)
    \item verringerte Verlässlichkeit je stärker gecachte Daten von tatsächlichen Daten abweichen
  \end{itemize}
\end{itemize}
Die REST Architektur baut auf der weit verbreiteten Client-Server Architektur auf.
REST ist zustandslos, sodass jede Anfrage vom Client zum Server alle für deren Verarbeitung notwendigen Informationen beinhalten muss.
Ein Client speichert gewöhnlich seinen Zustand, während auf Serverseite keine Zustandsinformationen über mehrere Anfragen hinweg gespeichert werden, wie es zum Beispiel bei Sessions der Fall ist.
Aufgrund dieser Eigenschaft wird im Zusammenhang mit REST auch von einer \textcquote[83]{REST}{Client-Cache-Stateless-Server[-Architektur]} gesprochen.
Hieraus ergeben sich drei Vorteile:\cite[vgl.][79]{REST}
\begin{itemize}
  \item Sichtbarkeit:
  Jede Anfrage kann separat untersucht werden, ohne weitere Informationen zu benötigen.
  \item Zuverlässigkeit:
  Anfragen können atomar betrachtet und bei Teilfehlern kann der Systemzustand leicht wiederhergestellt werden.
  \item Skalierbarkeit:
  Da ein Server Zustandsinformationen nur für die Dauer der Verarbeitung einer Anfrage speichert, können Ressourcen schnell wieder freigegeben werden.
\end{itemize}
Nachteil dieser Anforderung ist eine verringerte Netzwerkperformance, da bei sequenziellen Anfragen an einen Server, Daten zur Clientidentifikation, Authentifizierung oder aus vorangegangene Antworten erneut gesendet werden müssen.
Die verringerte Kopplung führt ebenfalls dazu, dass die Kontrolle des Servers über das Verhalten der Clientanwendung reduziert wird.
\par
Clientseitiges Speichern von erhaltenen Daten (Caching) erlaubt die Wiederverwendung früherer Serverantworten für zukünftige, gleiche Anfragen und ist sinnvoll, da die gefühlte Performance durch Verringerung der durchschnittlichen Latenz erhöht wird.
Dadurch wird die Effizienz und Skalierbarkeit der Anwendung erhöht, obwohl jede einzelne Latenz durch den Cache-Lookup erhöht wird.
Je mehr die gespeicherten Daten von den tatsächlichen Daten abweichen, desto mehr verringert sich die Verlässlichkeit derselben.\cite[vgl.][80]{REST}
GET-Requests werden von Webbrowsern implizit gecacht, wobei Server das Caching explizit erlauben und verbieten können.

\begin{itemize}
  \item 4 Grundsätze für Komponentenschnittstellen
  \begin{itemize}
    \item Identifizierung von Ressourcen
    \item Manipulation von Ressourcen durch ihre Repräsentation
    \item Selbsterklärende Nachrichten
    \item `hypermedia as the engine of application state'
  \end{itemize}
  \item Die Daten werden zum Ort der Verarbeitung geschickt, nicht die Anweisungen zu den Daten.
  \item Komponenten in REST Architektur sehen nur Komponenten, mit denen sie direkt interagieren (Schichten). Begrenztes Wissen verringert Komplexität. Schichten ermöglichen Kapselung. Zwischenkomponenten (Proxies) können Daten transformieren (wie Pipes/Filter)
  \item Schichtenarchitektur bedeutet mehr Datenverarbeitung und Latenz
  \item Client kann Repräsentation der Daten wählen. Ursprung der Daten hinter Serverinterface versteckt
  \item Ressourcen ist Abstraktion für jede Art Information (Dokumente, Bilder, Sammlung anderer Ressourcen)
  \item Jede Webseite ist Ressource
  \item Ressource wird durch Identifier (Bezeichner) bekannt gemacht und abrufbar. Bezeichner ändert sich nicht, wenn sich die Ressource ändert. Mehrere Ressourcen können die gleichen Informationen beinhalten.
  \item Repräsentation einer Ressource (Antwort des Servers) besteht aus Daten und Metadaten
  \item Kontrolldaten übermitteln den Zweck der Nachricht oder zum Umgang mit der Nachricht (HTTP Methoden, Status codes, Header)
  \item REST kennt drei Komponententypen:
  \begin{itemize}
    \item user agent: Web Browser, Benutzeranwendung, letztendlicher Empfänger der Antwort
    \item origin server: endgültige Quelle der Repräsentation der Ressource, letztendlicher Empfänger von Requests, die Modifikationen vornehmen; bietet Schnittstelle als Hierarchie von Ressourcen
    \item intermediary: agiert sowohl als Client, als auch als Server; leitet Requests und Responses weiter bzw.\ modifiziert sie; Gateway oder Proxy;
  \end{itemize}
  \item Leichtes Einführen von Zwischenkomponenten möglich durch selbstbeschreibende Nachrichten, generische Client- und Server-Schnittstellen und zustandslose Kommunikation. Keine einzige Komponente braucht Überblick über ganzes System.
  \item REST ermöglicht Verbindung zu anderen nicht-REST Systemen, indem diese eine REST-konforme Schnittstelle bereitstellen.
  \item Abbildung REST connectors and components
\end{itemize}
REST beschreibt drei Typen von Komponenten:\cite[vgl.][96]{REST}
\begin{itemize}
  \item User Agent:
  Dies kann ein Webbrowser bzw.\ die Benutzeranwendung sein, in jedem Fall jedoch der letztendliche Empfänger der Antwort.
  \item Origin Server:
  Die endgültige Quelle der Repräsentation einer Ressource und der letztendliche Empfänger von Request, die zu Modifikationen der Daten führen.
  \item Intermediary:
  Diese befinden sich zwischen User Agent und Origin Server und können sowohl als Client, als auch als Server agieren. Sie leiten Anfragen und Antworten weiter, können diese aber auch modifizieren. Je nach Anwendungsfall spricht man hier von einem Gateway oder Proxy.
\end{itemize}

\begin{figure}[h]
  \centering
  \includegraphics[width=\linewidth]{../Kapitel/Grafiken/REST-dissertation.png}
  \caption{REST}
  \label{img:REST-diss}
\end{figure}

\section{Implementierung von RESTful APIs}
\begin{itemize}
  \item APIs die dem REST Archtekturstil folgen werden RESTful genannt
  \item Richardson Maturity Model ermöglicht Bestimmung wie REST konform Web service (API) ist
  \item Level 0
  \item Level 1
  URI
  \item Level 2
  \begin{itemize}
    \item HTTP Methoden genutzt als Kontrolldaten um Intention auszudrücken
    \item CRUD Operationen werden abgedeckt
    \item GET:\ Anfragen einer Repräsentation der Ressource
    \item POST:\ kann zum Erstellen, Modifizieren und Löschen von Ressourcen verwendet werden, schlecht definiert; Funktionsweise in folgende Methoden aufgeteilt
    \item PUT:\ Erstellen/Ersetzen einer Repräsentation
    \item PATCH:\ Modifizieren einer Repräsentation
    \item DELETE:\ Löschen der Ressource
    \item GET, PUT, DELETE sind idempotent (gleiches Ergebnis bei mehrmaliger Ausführung). GET ist safe (kein Verändern der Ressource).
  \end{itemize}
  \item Level 3
  \begin{itemize}
    \item Hypermedia ermöglicht Navigation durch die API\@. Client ändert seinen Zustand, indem er URIs (Links) folgt (HATEOAS)
    \item keine externe Dokumentation nötig. Links zwischen Dokumenten dokumentieren die Ressourcen
    \item Datenformat ist entscheidend. Bestimmte Formate haben native Unterstützung für Links und Forms (HTML, ATOM)
    \item Media Type bestimmt Auswertung (und Anzeige) der Antwort. JSON, XML können genutzt werden. Client benötigt Informationen über Datenstruktur, um Links in diesen Dokumenten auszuwerten.
  \end{itemize}
\end{itemize}

\section{Verbreitung und Standardisierung}
\begin{itemize}
  \item REST bestimmt nicht welches Format benutzt werden muss.
  \item Kein REST Standard
  \item REST APIs nutzen Web Standards (HTTP, URI, Hypermedia)
  \item XML und HTML zur direkten Anzeige geeignet. JSON beliebter geworden, das leichter für Menschen und Maschinen zu lesen
  \item verschiedene Ansätze um Struktur von JSON Dokumenten zur Verwendung in APIs zu definieren. Teilweise miteinander verwendbar (definieren verschiedene Aspekte der Kommunikation)
  \item OpenAPI
  \item JSON:API
  \item Abbildung Beispiel Request und Response
\end{itemize}
