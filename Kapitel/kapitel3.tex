\section{Das REST Architekturkonzept}
\subsection{Entstehung}
\begin{itemize}
  \item Bekanntmachung Roy T. Fielding in Dissertation 2000
  \item Akronym für Representational State Transfer
  \item Prinzipien für die Entwicklung von verteilten Systemen. Baut auf bekannten Architekturen auf (Client-Server)
\end{itemize}

\subsection{Grundlagen}
\begin{itemize}
  \item Ziele von REST
  \begin{itemize}
    \item skalierbare Komponenteninteraktionen
    \item generische Interfaces
    \item unabhängige Entwicklung der Komponenten
    \item Zwischenkomponenten können spezielle Aufgaben übernehmen (Cache, Sicherheit, Schnittstelle zu Altsystemen)
  \end{itemize}
  \item REST hatte Einfluss (und macht Gebrauch von) HTTP und URI (IRI)
  \item REST ist zustandslos (daher auch client-stateless-server). Jeder Request von Client zu Server beinhaltet alle notwendigen Informationen, um den Request zu verstehen.
  \item Client speichert gewöhnlich Zustand. Server behält keine Clientsession bei.
  \item Vorteile
  \begin{itemize}
    \item Sichtbarkeit: Requests können einzeln untersucht werden
    \item Zuverlässigkeit: einfachere Wiederherstellung bei teilweisen Fehlern
    \item Skalierbarkeit: Server kann schnell Ressourcen wieder freigeben. Speichert keine Zustände
  \end{itemize}
  \item Nachteile
  \begin{itemize}
    \item verringerte Netzwerkperformance, da mit jedem Request Daten wiederholt
    \item verringerte Kontrolle des Servers über Verhalten der Clientanwendung
  \end{itemize}
  \item Clientseitiges Caching
  \begin{itemize}
    \item Ressourcen implizit oder explizit gecacht
    \item erlaubt Wiederverwendung früherer Serverantworten für zukünftige, gleiche Requests
    \item bessere Effizienz und Skalierbarkeit, erhöhte gefühlte Performance durch Verringerung der durchschnittlichen Latenz (jede einzelne Latenz durch Cache-Lookup erhöht)
    \item verringerte Verlässlichkeit je stärker gecachte Daten von tatsächlichen Daten abweichen
  \end{itemize}
  \item 4 Grundsätze für Komponentenschnittstellen
  \begin{itemize}
    \item Identifizierung von Ressourcen
    \item Manipulation von Ressourcen durch ihre Repräsentation
    \item Selbsterklärende Nachrichten
    \item `hypermedia as the engine of application state'
  \end{itemize}
  \item Die Daten werden zum Ort der Verarbeitung geschickt, nicht die Anweisungen zu den Daten.
  \item Komponenten in REST Architektur sehen nur Komponenten, mit denen sie direkt interagieren (Schichten). Begrenztes Wissen verringert Komplexität. Schichten ermöglichen Kapselung. Zwischenkomponenten (Proxies) können Daten transformieren (wie Pipes/Filter)
  \item Schichtenarchitektur bedeutet mehr Datenverarbeitung und Latenz
  \item Client kann Repräsentation der Daten wählen. Ursprung der Daten hinter Serverinterface versteckt
  \item Ressourcen ist Abstraktion für jede Art Information (Dokumente, Bilder, Sammlung anderer Ressourcen)
  \item Jede Webseite ist Ressource
  \item Ressource wird durch Identifier (Bezeichner) bekannt gemacht und abrufbar. Bezeichner ändert sich nicht, wenn sich die Ressource ändert. Mehrere Ressourcen können die gleichen Informationen beinhalten.
  \item Repräsentation einer Ressource (Antwort des Servers) besteht aus Daten und Metadaten
  \item Kontrolldaten übermitteln den Zweck der Nachricht oder zum Umgang mit der Nachricht (HTTP Methoden, Status codes, Header)
  \item REST kennt drei Komponententypen:
  \begin{itemize}
    \item user agent: Web Browser, Benutzeranwendung, letztendlicher Empfänger der Antwort
    \item origin server: endgültige Quelle der Repräsentation der Ressource, letztendlicher Empfänger von Requests, die Modifikationen vornehmen; bietet Schnittstelle als Hierarchie von Ressourcen
    \item intermediary: agiert sowohl als Client, als auch als Server; leitet Requests und Responses weiter bzw.\ modifiziert sie; Gateway oder Proxy;
  \end{itemize}
  \item Leichtes Einführen von Zwischenkomponenten möglich durch selbstbeschreibende Nachrichten, generische Client- und Server-Schnittstellen und zustandslose Kommunikation. Keine einzige Komponente braucht Überblick über ganzes System.
  \item REST ermöglicht Verbindung zu anderen nicht-REST Systemen, indem diese eine REST-konforme Schnittstelle bereitstellen.
  \item Abbildung REST connectors and components
\end{itemize}

\subsection{Implementierung von RESTful APIs}
\begin{itemize}
  \item APIs die sich an REST halten werden RESTful genannt
  \item Richardson Maturity Model ermöglicht Bestimmung wie REST konform Web service (API) ist
  \item Level 0
  \item Level 1
  URI
  \item Level 2
  \begin{itemize}
    \item HTTP Methoden genutzt als Kontrolldaten um Intention auszudrücken
    \item CRUD Operationen werden abgedeckt
    \item GET:\ Anfragen einer Repräsentation der Ressource, idempotent
    \item POST:\ kann zum Erstellen, Modifizieren und Löschen von Ressourcen verwendet werden, schlecht definiert; Funktionsweise in folgende Methoden aufgeteilt
    \item PUT:\ Erstellen/Ersetzen einer Repräsentation
    \item PATCH:\ Modifizieren einer Repräsentation
    \item DELETE:\ Löschen der Ressource
  \end{itemize}
  \item Level 3
  \begin{itemize}
    \item Hypermedia ermöglicht Navigation durch die API\@. Client ändert seinen Zustand, indem er URIs (Links) folgt (HATEOAS)
    \item keine externe Dokumentation nötig. Links zwischen Dokumenten dokumentieren die Ressourcen
    \item Datenformat ist entscheidend. Bestimmte Formate haben native Unterstützung für Links und Forms (HTML, ATOM)
    \item Media Type bestimmt Auswertung (und Anzeige) der Antwort. JSON, XML können genutzt werden. Client benötigt Informationen, um Links in diesen Dokumenten auszuwerten.
  \end{itemize}
\end{itemize}

\subsection{Verbreitung und Standardisierung}
\begin{itemize}
  \item REST bestimmt nicht welches Format benutzt werden muss.
  \item Kein REST Standard
  \item XML und HTML zur direkten Anzeige geeignet. JSON beliebter geworden, das leichter für Menschen und Maschinen zu lesen
  \item verschiedene Ansätze um Struktur von JSON Dokumenten zur Verwendung in APIs zu definieren. Teilweise miteinander verwendbar (definieren verschiedene Aspekte der Kommunikation)
  \item OpenAPI
  \item JSON:API
  \item Abbildung Beispiel Request und Response
\end{itemize}
