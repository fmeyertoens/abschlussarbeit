\section{Einleitung}

\subsection{Motivation und Zielstellung}
\begin{itemize}
  \item Entwicklung von Web Anwendungen über die Zeit vom Monolith zu Service-orientierter Architektur
  \item Single Page Applikationen gesamte Kommunikation über Web APIs
  \item Vielzahl von internen Services im Unternehmen und externen Serviceanbietern
\end{itemize}
\subsection{Aufbau der Arbeit}
\begin{itemize}
  \item Betrachtung der Entwicklung von Web Anwendungen mit der Client-Server Architektur als Grundlage
  \item Abgrenzung des Begriffes API und Differenzierung von anderen API Ansätzen
  \item Das REST Architekturkonzept als Grundlage für das Web und APIs
  \item GraphQL als Alternative, seine Funktionsweise
  \item Vergleich von REST und GraphQL
  \item Welchen klassischen Problemen müssen sich API Entwickler stellen?
  \item Welche Probleme von REST löst GraphQL\@?
  \item Welche Vorteile hat REST gegenüber GraphQL\@?
  \item Vorstellung einer Auswahl von Tools und Bibliotheken, die verschiedene Probleme von REST und GraphQL lösen bzw.\ die Entwicklung vereinfachen.
  \item Untersuchung der Kompatibilität von Bibliotheken
  \item Tests von REST und GraphQL in verschiedenen Szenarien.
  \item Kombinierter Einsatz von GraphQL und REST
\end{itemize}