\section{Vorbetrachtungen}
\subsection{Client-Server Architektur}
\begin{itemize}
  \item Client-Server ist ein verteiltes System
  \item Zwischen Client und Server geschieht Nachrichtenaustausch
  \item Client fordert eine Operation vom Server an. Server sendet Resultat der Operation an den Client zurück.
  \item Client initiiert die Interaktion. Server reagiert.
  \item Mehrere Clients können den gleichen Server nutzen. Abbildung aus `Grundkurs verteilte Systeme'!
  \item Client kann mehrere Server benutzen. Server kann in anderer Interaktion selbst zum Client/Vermittler werden.
  \item Vorteile
    \begin{itemize}
      \item getrennte Entwicklung
      \item unabhängige Ausfälle
      \item Festgelegte Rollenverteilung: Client ist Konsument. Server ist Produzent.
    \end{itemize}
  \item Herausforderung: einheitliches Kommunikationsprotokoll
\end{itemize}